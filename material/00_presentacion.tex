% Created 2026-02-02 Mon 22:23
% Intended LaTeX compiler: pdflatex
\documentclass[presentation]{beamer}
\usepackage[utf8]{inputenc}
\usepackage[T1]{fontenc}
\usepackage{graphicx}
\usepackage{longtable}
\usepackage{wrapfig}
\usepackage{rotating}
\usepackage[normalem]{ulem}
\usepackage{amsmath}
\usepackage{amssymb}
\usepackage{capt-of}
\usepackage{hyperref}
\setbeamercolor{background canvas}{bg=}
\usepackage[absolute,overlay]{textpos}
\usepackage{yfonts}
\usepackage{tikz}
\usepackage[normalem]{ulem}
\definecolor{UnamBlue}{RGB}{3,40,109}
\definecolor{gris1}{RGB}{40,40,40}
\definecolor{gris2}{RGB}{80,80,80}
\definecolor{gris3}{RGB}{120,120,120}
\setbeamercolor{alerted text}{fg=red}
\setbeamertemplate{footline}[slide number]
\definecolor{string}{rgb}{0,0.6,0} \definecolor{shadow}{rgb}{0.5,0.5,0.5} \definecolor{keyword}{rgb}{0.58,0,0.82} \definecolor{identifier}{rgb}{0,0,0.7}
\setbeamerfont{author}{size=\footnotesize}
\newcommand{\attrib}[3]{ \begin{textblock*}{150mm}[1,1](125mm,93mm)  {\tiny \hfill \textcolor{gray}{\href{#3}{#1 \textcolor{lightgray}{(#2)}}}} \end{textblock*} }
\newcommand{\img}[4]{ \includegraphics[width=\textwidth]{#1} \attrib{#2}{#3}{#4} }
\newcommand*\circulo[1]{\tikz[baseline=(char.base)]{ \node[shape=circle,draw,inner sep=2pt] (char) {#1};}}
\graphicspath{ {./img/} }
\newcommand{\Smiley}{\tikz[baseline=-0.75ex]{ \draw circle (2mm); \node[fill,circle,inner sep=0.5pt] (left eye) at (135:0.8mm) {}; \node[fill,circle,inner sep=0.5pt] (right eye) at (45:0.8mm) {}; \draw (-145:0.9mm) arc (-120:-60:1.5mm); } }
\newcommand{\frownie}{\tikz[baseline=-0.75ex]{\draw circle (2mm); \node[fill,circle,inner sep=0.5pt] (left eye) at (135:0.8mm) {}; \node[fill,circle,inner sep=0.5pt] (right eye) at (45:0.8mm) {}; \draw (-145:0.9mm) arc (120:60:1.5mm); } }
\newcommand{\neutranie}{\tikz[baseline=-0.75ex]{\draw circle (2mm); \node[fill,circle,inner sep=0.5pt] (left eye) at (135:0.8mm) {}; \node[fill,circle,inner sep=0.5pt] (right eye) at (45:0.8mm) {}; \draw (-135:0.9mm) -- (-45:0.9mm); } }
\AtBeginSection[]{ \begin{frame}<beamer> \frametitle{Índice} \tableofcontents[currentsection] \end{frame} }
\usetheme{metropolis}
\usecolortheme{lily}
\author{Gunnar Wolf}
\date{}
\title{Presentación del curso}
\subtitle{Fundamentos y Técnicas de Seguridad para Aplicaciones}
\hypersetup{
 pdfauthor={Gunnar Wolf},
 pdftitle={Presentación del curso},
 pdfkeywords={},
 pdfsubject={},
 pdfcreator={Emacs 30.1 (Org mode 9.7.11)}, 
 pdflang={English}}
\begin{document}

\maketitle
\section{Punto de partida}
\label{sec:org4bd1e34}

\begin{frame}[label={sec:org42019b9}]{¡Hola! Soy Gunnar, y soy programador}
\centering
\begin{center}
\includegraphics[width=0.9\textwidth]{./gwolf_librero.jpg}
\end{center}
\pause \scriptsize

(Aquí es cuando todos dicen \emph{``¡Hola Gunnar!''})
\end{frame}
\begin{frame}[label={sec:org2404478}]{Mis coordenadas}
\centering Datos personales:

\begin{description}
\item[{Nombre}] Gunnar Eyal Wolf Iszaevich
\item[{Email}] gwolf@gwolf.org
\item[{Ubicación}] Instituto de Investigaciones Económicas UNAM (Secretaría
Técnica)
\item[{Teléfono}] 55-5623-0154
\end{description}

\vskip 1.5em De este curso:

\begin{description}
\item[{Repositorio Git}] { \small\url{https://github.com/gwolf/seg_aplic_2026-2} }

\pause \vskip 2em \scriptsize \raggedleft
Importante: \textbf{¿Cómo se llevan con Git?}
\end{description}
\end{frame}
\begin{frame}[label={sec:orge8f0929}]{Horario / calendario}
\begin{itemize}
\item Martes y jueves, 8:30–10:00 (3hr semanales)
\item Del 3 de febrero al 29 de mayo de 2026 (16 semanas)
\item \alert{48} horas previstas efectivas de cursado
\end{itemize}
\end{frame}
\section{El cursado: Revisión temática}
\label{sec:orgaf82ab8}

\begin{frame}[label={sec:orgedaafc5}]{Generalidades}
\begin{description}
\item[{Campo de conocimiento}] Redes y seguridad en cómputo
\item[{Semestre}] 2°–3°
\item[{Objetivo general}] Que el alumno identifique las principales amenazas a
la seguridad informática, y conozca las principales técnicas para
reconocerlas y evitarlas
\item[{Unidades}] \begin{enumerate}
\item Introducción a la seguridad de la información
\item El panorama de la seguridad informática hoy
\item Tipos de ataque
\item Prácticas de programación segura
\end{enumerate}
\end{description}
\end{frame}
\begin{frame}[label={sec:orgd68bd6e},fragile]{Ojo: Plan de estudios \emph{flexible y adaptable}}
 \begin{itemize}
\item Por el \emph{nivel} en el que estamos, y por el \emph{tamaño del grupo} que
tenemos, los temas a desarrollar \alert{dependen de sus intereses y
habilidades}
\item Presento una \emph{propuesta} de plan de estudios
\item Pero los temas \emph{deben seguir} \alert{sus temas prioritarios}
\begin{itemize}
\item \(\Rightarrow\) Su experiencia personal/profesional
\item \(\Rightarrow\) Sus proyectos de tesis
\end{itemize}
\end{itemize}
\vskip 2em \pause \begin{center}
\alert{¡Primera (micro)tarea!}

En el repositorio Git, bajo \texttt{/entregas/intereses/<ApellidoNombre>/},
escriban sus intereses y expectativas para la materia.
\end{center}
\end{frame}
\begin{frame}[label={sec:orgcc81d3b}]{1. Introducción a la seguridad de la información}
\begin{columns} \begin{column}{0.3\textwidth}
\begin{center}
\includegraphics[width=\textwidth]{./candados.jpg}
\end{center}
\end{column} \begin{column}{0.7\textwidth}
\begin{enumerate}
\item ¿Qué es seguridad de la información?
\item Características de un sistema seguro
\item Propiedades de la seguridad de la información
\item Seguridad informática capa a capa
\item Evolución de los retos de seguridad informática sobre el desarrollo histórico de la computación
\end{enumerate}
\end{column} \end{columns}

\vskip 1em \raggedleft \scriptsize Imagen: propia
\end{frame}
\begin{frame}[label={sec:org0f6c3b8}]{2. El panorama de la seguridad informática hoy}
\begin{columns} \begin{column}{0.7\textwidth}
\begin{enumerate}
\item Modelos de riesgo / Modelos de ataque
\item Manejo y respuesta a vulnerabilidades
\item Gestión de permisos y privilegios
\begin{itemize}
\item Principio del menor privilegio
\end{itemize}
\item Bases de datos de vulnerabilidades y debilidades
\begin{itemize}
\item CVE — Common Vulnerabilities and Exposure
\item CWE — Common Weakness Enumeration
\end{itemize}
\end{enumerate}
\end{column} \begin{column}{0.3\textwidth}
\begin{center}
\includegraphics[width=\textwidth]{./tagcloud.png}
\end{center}
\end{column} \end{columns}

\vskip 1em \raggedleft \scriptsize Imagen: TechNewsRadio: Updated CISSP Domains \\ \url{https://www.technewsradio.com/2015/03/updated-cissp-domains.html}
\end{frame}
\begin{frame}[label={sec:org5094d7d}]{3. Tipos de ataque}
\begin{columns} \begin{column}{0.4\textwidth}
\begin{center}
\includegraphics[width=\textwidth]{./attacker.jpg}
\end{center}
\end{column} \begin{column}{0.6\textwidth}
\begin{enumerate}
\item Denegación de servicio (DoS, DDoS, amplificación, reflexión)
\item Desbordamientos (de buffer, de stack, de enteros)
\item Inyecciones (de cadenas, SQL, objetos, código, solicitudes)
\item Cruce de límites de confianza (XSS, CSRF)
\item Basados en codificación
\item Algoritmos débiles de cifrado
\end{enumerate}
\end{column} \end{columns}

\vskip 1em \raggedleft \scriptsize Imagen: Increasing Your Digital Responsibility - Protect Yourself   \url{https://wecdsbit.blogspot.com/2017/05/increasing-your-digital-responsibility.html}
\end{frame}
\begin{frame}[label={sec:org85bb42a}]{4. Prácticas de programación segura}
\begin{columns} \begin{column}{0.7\textwidth}
\begin{enumerate}
\item Principios de diseño seguro
\item Prácticas seguras de programación
\begin{itemize}
\item Estándares y convenciones
\item Manejo seguro de datos
\item Manejo de errores
\item Manejo de información sensible (contraseñas / datos personales)
\end{itemize}
\item Evolución de los lenguajes de programación en lo relativo a la seguridad
\item Consideraciones de versionado y mantenibilidad
\end{enumerate}
\end{column} \begin{column}{0.3\textwidth}
\begin{center}
\includegraphics[width=\textwidth]{./consejos_seguridad.jpg}
\end{center}
\vskip 6em \, \end{column} \end{columns}

\vskip 1em \raggedleft \scriptsize Imagen: Seguridad en internet – 4 consejos de seguridad básicos para tu sitio Web \url{https://www.socialfuturo.com/internet/seguridad-en-internet-4-consejos-de-seguridad-basicas-para-tu-sitio-web/}
\end{frame}
\section{¿Cómo evaluaremos?}
\label{sec:org3d2ae7d}
\begin{frame}[label={sec:org2e30117}]{¿En qué consistirá la evaluación?}
\begin{itemize}
\item No voy a mentirles: \emph{¡No lo sé!} \neutranie
\item Es la \sout{primera} segunda vez que propongo un curso a nivel posgrado\ldots{} ¡No
crean que no impone un poco! \Smiley
\begin{itemize}
\item La evaluación tiene que ser mucho más \emph{cercana} que en un curso de
licenciatura
\end{itemize}
\item \ldots{} ¡Pero me comprometo a ser justo y a no perjudicarlos!
\item Ya estando en nivel posgrado, creo que podremos ir acordando eso en
conjunto
\end{itemize}
\begin{center}
\includegraphics[height=0.4\textheight]{./cuaderno_eval.jpg}
\end{center}
\vskip 1em \raggedleft \scriptsize Rúbrica para evaluar mapas mentales – Nivel intermedio \url{https://gesvinromero.com/2016/02/18/rubrica-para-evaluar-mapas-mentales-nivel-intermedio-ebook/}
\end{frame}
\begin{frame}[label={sec:orgd0ace42}]{Ideas para la evaluación}
\begin{itemize}
\item Exposición para debate de los temas que vayamos tocando
\item Tareas: Implementaciones (código) de algunos conceptos que abordemos
\begin{itemize}
\item \emph{Prácticas} de conceptos más simples que sea necesario reforzar
\end{itemize}
\item Algún tema más extenso a exponer, a propuesta de cada uno de ustedes
\end{itemize}
\end{frame}
\begin{frame}[label={sec:orgbfbbc82}]{El elefante en el cuarto: ¿LLMs / GenAI / GPT / como-le-digas?}
\begin{columns} \begin{column}{0.3\textwidth}
\begin{center}
\includegraphics[width=\textwidth]{./elefante.jpg}
\end{center}
\end{column} \begin{column}{0.7\textwidth}
\begin{itemize}
\item No podemos cerrarnos a su existencia (¡y conveniencia!)
\item Pero tampoco podemos permitir que \emph{anulen} al esfuerzo en las consignas
\item Aprovechemos y exploremos — pero con \alert{honestidad académica}, usándolos
para \alert{mejorar} (no reemplazar) nuestro esfuerzo
\end{itemize}
\end{column} \end{columns}

\begin{columns} \begin{column}{0.7\textwidth}
\begin{itemize}
\item Todos somos nuevos usándolos, y tenemos que aprender
\item Algunos somos más bien \emph{escépticos}. ¡Pero no hay que cerrarnos! \Smiley
\end{itemize}
\end{column} \begin{column}{0.3\textwidth}
\begin{center}
\includegraphics[width=\textwidth]{./elefante2.jpg}
\end{center}
\end{column} \end{columns}
\raggedleft \scriptsize Imágenes:  On 2012, Blogging and Elephants in the Room», \url{https://www.chroniclesofcardigan.com/}
\end{frame}
\begin{frame}[label={sec:orgf510e48}]{¿Algo más?}
¡Espero que estén tan emocionados como yo de iniciar este cursado!

\begin{description}
\item[{Nombre}] Gunnar Eyal Wolf Iszaevich
\item[{Email}] gwolf@gwolf.org
\item[{Ubicación}] Instituto de Investigaciones Económicas UNAM (Secretaría
Técnica)
\item[{Teléfono}] 55-5623-0154
\end{description}
\end{frame}
\end{document}
